\usepackage[match,haranoaji]{luatexja-preset} % 日本語(HaranoAji)を一括設定
\usepackage{unicode-math}
\setmathfont{TeX Gyre Termes Math}
\usepackage[margin=25mm]{geometry}
\usepackage{hyperref}

\usepackage{emoji}                 % 絵文字
\setemojifont{Apple Color Emoji}   % macOS のカラーフォントを明示(任意)
\usepackage{newunicodechar}
\newunicodechar{😀}{\emoji{grinning-face}} % ← ここがポイント(1f600ではなく名前)
\newunicodechar{⁰}{\textsuperscript{0}}
\newunicodechar{¹}{\textsuperscript{1}}
\newunicodechar{²}{\textsuperscript{2}}
\newunicodechar{³}{\textsuperscript{3}}
\newunicodechar{⁴}{\textsuperscript{4}}
\newunicodechar{⁵}{\textsuperscript{5}}
\newunicodechar{⁶}{\textsuperscript{6}}
\newunicodechar{⁷}{\textsuperscript{7}}
\newunicodechar{⁸}{\textsuperscript{8}}
\newunicodechar{⁹}{\textsuperscript{9}}
\newunicodechar{⁺}{\textsuperscript{+}}
\newunicodechar{⁻}{\textsuperscript{-}}
\newunicodechar{⁼}{\textsuperscript{=}}
\newunicodechar{⁽}{\textsuperscript{(}}
\newunicodechar{⁾}{\textsuperscript{)}}
\newunicodechar{ⁿ}{\textsuperscript{n}}
\newunicodechar{ⁱ}{\textsuperscript{i}}

% --- subscripts (U+2080–U+2089, ほか) ---
\newunicodechar{₀}{\textsubscript{0}}
\newunicodechar{₁}{\textsubscript{1}}
\newunicodechar{₂}{\textsubscript{2}}
\newunicodechar{₃}{\textsubscript{3}}
\newunicodechar{₄}{\textsubscript{4}}
\newunicodechar{₅}{\textsubscript{5}}
\newunicodechar{₆}{\textsubscript{6}}
\newunicodechar{₇}{\textsubscript{7}}
\newunicodechar{₈}{\textsubscript{8}}
\newunicodechar{₉}{\textsubscript{9}}
\newunicodechar{₊}{\textsubscript{+}}
\newunicodechar{₋}{\textsubscript{-}}
\newunicodechar{₌}{\textsubscript{=}}
\newunicodechar{₍}{\textsubscript{(}}
\newunicodechar{₎}{\textsubscript{)}}
% よく使う下付きラテン文字
\newunicodechar{ₐ}{\textsubscript{a}}
\newunicodechar{ₑ}{\textsubscript{e}}
\newunicodechar{ₒ}{\textsubscript{o}}
\newunicodechar{ₓ}{\textsubscript{x}}
\newunicodechar{ₕ}{\textsubscript{h}}
\newunicodechar{ₖ}{\textsubscript{k}}
\newunicodechar{ₗ}{\textsubscript{l}}
\newunicodechar{ₘ}{\textsubscript{m}}
\newunicodechar{ₙ}{\textsubscript{n}}
\newunicodechar{ₚ}{\textsubscript{p}}
\newunicodechar{ₛ}{\textsubscript{s}}
\newunicodechar{ₜ}{\textsubscript{t}}

% header-quote-bg.tex
\usepackage{xcolor}
\usepackage[most]{tcolorbox}
\let\oldquote\quote
\let\endoldquote\endquote

% 背景色だけ(枠なし)で引用を目立たせる
\definecolor{quoteBG}{HTML}{F2F7FF} % 青みの淡色。茶系なら D7CCC8 などに変更

\renewenvironment{quote}{\quotation}
  {\endquotation}
\renewenvironment{quotation}
  {\begin{tcolorbox}[enhanced,breakable,boxrule=0pt,frame hidden,arc=0pt,outer arc=0pt,
                     colback=quoteBG,left=12pt,right=12pt,top=18pt,bottom=12pt]}
  {\end{tcolorbox}}

% ChangeLog
% 2025-10-23 v1.0 初版作成 v1.1, v1.2の変更は削除
% 2025-10-23 v1.3 quote環境をquotation環境に変更。quotation側でtcolorboxを使うように修正
